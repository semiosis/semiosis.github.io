% Created 2024-10-19 Sat 11:02
% Intended LaTeX compiler: xelatex
\documentclass[11pt]{article}
\usepackage[mathletters]{ucs}
\usepackage{graphicx}
\usepackage{longtable}
\usepackage{wrapfig}
\usepackage{rotating}
\usepackage[normalem]{ulem}
\usepackage{amsmath}
\usepackage{amssymb}
\usepackage{capt-of}
\usepackage{hyperref}
\usepackage[margin=0.5in]{geometry}
\author{Shane Mulligan}
\date{\textit{<2024-10-19 Sat>}}
\title{Greek study}
\hypersetup{
 pdfauthor={Shane Mulligan},
 pdftitle={Greek study},
 pdfkeywords={faith christianity},
 pdfsubject={},
 pdfcreator={Emacs 29.1.50 (Org mode 9.6.8)}, 
 pdflang={English}}
\begin{document}

\maketitle

\section{NT Greek study}
\label{sec:org9b0acc9}
\subsection{Case}
\label{sec:org5f359e9}

\begin{verbatim}
 1  interrogative
 2      /ˌɪntəˈrɒɡətɪv/
 3      adjective
 4      having the force of a question.
 5      "a hard, interrogative stare"
 6  
 7  transitive verb
 8      A verb that entails one or more transitive
 9      objects, for example, 'enjoys' in Amadeus
10      enjoys music.
11  
12      This contrasts with intransitive verbs,
13      which do not entail transitive objects,
14      for example, 'arose' in Beatrice arose.
\end{verbatim}

Notes from \url{https://en.wikipedia.org/wiki/Grammatical\_case}:

\begin{itemize}
\item \textbf{N (Nominative)}
\begin{itemize}
\item \texttt{Indicates}: Subject of a finite verb
\item \texttt{Sample case words}: we
\item \texttt{Sample sentence}: \uline{We} went to the store.
\item \texttt{Interrogative}: Who or what?
\item \texttt{Notes}: Corresponds to English's subject pronouns.
\end{itemize}
\item \textbf{V (Vocative)}
\begin{itemize}
\item \texttt{Indicates}: Addressee
\item \texttt{Sample case words}: John
\item \texttt{Sample sentence}:
\begin{itemize}
\item \uline{John}, are you all right?
\item Hello, \uline{John}!
\item O John, how are you! (archaic)
\end{itemize}
\item \texttt{Interrogative}:
\item \texttt{Notes}: Roughly corresponds to the archaic use of "O" in English.
\end{itemize}
\item \textbf{A (Accusative)}
\begin{itemize}
\item \texttt{Indicates}: Direct object of a transitive verb
\item \texttt{Sample case words}: us, for us, the (object)
\item \texttt{Sample sentence}:
\begin{itemize}
\item The clerk remembered \uline{us}.
\item John waited \uline{for us} at the bus stop.
\item Obey \uline{the law}.
\end{itemize}
\item \texttt{Interrogative}: 
\begin{itemize}
\item Whom or what?
\end{itemize}
\item \texttt{Notes}: Corresponds to English's object pronouns and preposition for construction before the object, often marked by a definite article the. Together with dative, it forms modern English's oblique case.
\end{itemize}
\item \textbf{G (Genitive)}
\begin{itemize}
\item \texttt{Indicates}:
\item \texttt{Sample case words}:
\item \texttt{Sample sentence}:
\item \texttt{Interrogative}:
\item \texttt{Notes}:
\end{itemize}
\item \textbf{D (Dative)}
\begin{itemize}
\item \texttt{Indicates}:
\item \texttt{Sample case words}:
\item \texttt{Sample sentence}:
\item \texttt{Interrogative}:
\item \texttt{Notes}:
\end{itemize}
\end{itemize}

\subsubsection{Example}
\label{sec:orge3c61c5}
\url{https://biblehub.com/interlinear/2\_john/1-9.htm}

\begin{verbatim}
1  tou
2  3588
3  tou
4  τοῦ
5   -
6  Art-GMS
\end{verbatim}

\begin{itemize}
\item G - \href{https://en.wikipedia.org/wiki/Grammatical\_case}{Genitive}
\item M - \href{https://en.wikipedia.org/wiki/Grammatical\_case}{Masculine}
\item S - \href{https://en.wikipedia.org/wiki/Grammatical\_case}{Singular}
\end{itemize}
\end{document}